\documentclass[10pt]{article}

%%%%%%%%%%%%%%%%%%%%%
% Package Inclusion %
%%%%%%%%%%%%%%%%%%%%%
\usepackage{geometry,amsmath,amsthm,mathrsfs,amssymb,graphicx,bm,hyperref,url}

%%%%%%%%%%%%%%%%%%%
% Custom Commands %
%%%%%%%%%%%%%%%%%%%
\newcommand{\n}{\noindent}
\newcommand{\norm}[1]{\left|#1\right|}
\newcommand{\avg}[1]{\left<#1\right>}

%%%%%%%%%%%%%%%%%%%%%%%%%%
% Title Page Information %
%%%%%%%%%%%%%%%%%%%%%%%%%%

\title{Notes for PHYS 232: Stellar Structure}
\author{Bill Wolf}
\date{\today}

\begin{document}

\vfill\maketitle\vfill \newpage

\tableofcontents \newpage

%%%%%%%%%%%%%%%%%%%%%%
% January 9, 2012 %
%%%%%%%%%%%%%%%%%%%%%%

\section{Introduction}
	\emph{Monday, January 9, 2012}\\
	
	\n In this course, we will make ample use of the new computational tool MESA: \textbf{M}odules for \textbf{E}xperiments in \textbf{S}teller \textbf{A}strophysics. Throughout the class, we will have small projects to get us accustomed to using this tool, culminating in an independent research project using MESA.

%%%%%
\section{Hydrostatics and Thermodynamics of Self-Gravitating Objects}
	\subsection{The HR Diagram} 
	Most stars shine predominantly in the optical. Thus we get most of our information about stars by observing their optical output. When plotting a population of stars' luminosities against their surface temperature (color), we note a strong correlation between the two. As it turns out, the controlling parameter for these quantities is the mass of the star, at least while the star is on the \textbf{Main Sequence} (stars burning hydrogen to helium). The correlation between the mass of a main-sequence star and its luminosity is incredibly strong (see HR diagram examples).
	\subsection{Conditions for a Star on the HR Diagram}
	We are interested in knowing what defines the regime where a star can reside in $L,\,T_{\mathrm{eff}}=[L/(4\pi\sigma_{\mathrm{SB}}R^2)]^{1/4}$. Why, for example, is there a dynamical range in the luminosity spanning over six orders of magnitude, while only a range of a factor of about 5 in the effective temperature? To gain some perspective, we might observe the number of stars as a function of brightness. We organize these stars by their \textbf{spectral type} (a rough measure of how big the star is) and find their approximate \textbf{mass density} (the amount of mass contained in these stars per unit volume):
	\begin{center} 
	\begin{tabular}{l l}
		Spectral Type & $\rho\ (M_\odot/\mathrm{pc^3})$\\
		\hline
		O-B & 0.4\\
		A-F & 4\\
		G-M & 40\\
		WD's & 20
	\end{tabular}
	\end{center}
	We see that the big, bright stars form an exceedingly small portion of the amount of stellar mass in our galaxy. We will find that this is because large stars exhaust their fuel much more quickly than smaller stars, and thus live and die much faster. \\
	
	\subsubsection{Population I Stars}
	\n Consider the Milky Way. From Earth, the center of the galaxy is approximately 8.5 kpc away. The disk is approximately 100 pc wide.  We've observed that stars in the thin disk (commonly known as \textbf{Population I Stars}) are orbiting at the orbital velocity with a small amount of axial and radial motion. They are essentially dynamically cold and in nearly circular orbits. This is indeed where most of the \textbf{interstellar medium} (ISM) resides, causing much of the star formation in the galaxy. This region is also very metal rich. That is, compared to other parts of the universe, there is a much higher concentration of elements heavier than helium present. We will denote the mass fraction with $Z$, and in this region, we have $Z\sim 1-2\,\%$. These metals come from a previous generation of stars, who died in the past, giving off the metals we now have.\\
	
	\subsubsection{Population II Stars}
	 \textbf{Population II Stars} reside mostly in the spheroid in the center of the galaxy. These are older stars in regions where star formation is largely shut down. Typically they are metal poor, with metallicities as low as $Z=10^{-4}Z_\odot$. Kinematically, they are typically on radial orbits. We typically say that the globular clusters are part of this population. Sometimes these stars are seen passing through the disk at velocities comparable to the orbital velocities, and are quite peculiar due to their high velocities and unique spectra (due to the low metallicities).
	 
	 \subsection{The Isothermal, Plane Parallel Atmosphere}
	 Consider an atmosphere where the local acceleration due to gravity, $\mathbf{g}$ is constant in value and direction. The atmosphere is composed of an isothermal ideal gas with temperature $T$. We wish to find the distribution of particles in this atmosphere. In a strictly statistical sense, we would expect the energy distribution to be comparable to $e^-{E/kT}$. In our case, the energy of particles is linear in height, so we expect this probability to proportional to $e^{-mgh/kT}$.\\
	 
	 \n We will let $m_B\approx m_p$ be the baryon mass, $\mu$ be the mean molecular weight, and $\rho$ is the density in $\mathrm{g\,cm^{-3}}$. We suppose that the gas is in hydrostatic balance, so we have
	 \begin{equation}
	 	\label{ippa.1} \frac{dP}{dz}=-\rho g
	 \end{equation}
	 Combining this with the ideal gas law,
	 \begin{equation}
	 	\label{ippa.2} P=nkT
	 \end{equation}
	We find that
	\begin{equation}
		\label{ippa.3} kT\frac{dn}{dz}=m_p\mu ng
	\end{equation}
	which in turn gives us
	\begin{equation}
		\label{ippa.4} \frac{d\ln n}{dz}=-\frac{m_p\mu g}{kT}
	\end{equation}
	Solving this differential equation gives the expected result
	\begin{equation}
		\label{ippa.5} n(z)=n(0)\exp\left(-\frac{m_p\mu gz}{kT}\right)=n(0)\exp\left(-\frac{z}{h}\right)
	\end{equation}
	where we have defined the \textbf{scale height} $h\equiv kT/(\mu m-pg)$, which is the e-folding distance in number density. As it turns out, the scale height for earth's atmosphere is approximately 10 km. Comparing the scale height to the size of an object (say, Earth), gives us
	\begin{equation}
		\label{ippa.6} \frac{h}{R}=\frac{kT}{\mu m_p\frac{GM}{R^2}R}\sim \frac{v_{\mathrm{th}}^2}{v_{\mathrm{esc}}^2}
	\end{equation}
	For stars, we will find that $kT_c/m_p\sim GM/R$. For a star, we have
	\begin{equation}
		\label{ippa.7} \frac{h}{R}\sim \frac{T_{\mathrm{eff}}}{T_c}
	\end{equation}
	This tells us that the edges of stars are quite sharp-edged (their scale heights are very small compared to their radii). We can deduce a physical meaning for the scale height as being how far a particle needs to fall to gain an energy comparable to $kT$.\\
	
	\n Again returning to the ideal gas law,
	\begin{equation}
		\label{ippa.8} P=\frac{\rho}{\mu m_p}kT=nkT
	\end{equation}
	and the condition for hydrostatic equilibrium,
	\begin{equation}
		\label{ippa.9} dP=-\rho g\,dz,
	\end{equation}
	we integrate \eqref{ippa.9} from $z=z$ to $z\to \infty$:
	\begin{align}
		\label{ippa.10} P(\infty)-P(z)&=-\int_z^\infty \rho(z') g\,dz'\\
		\label{ippa.11} P(z) &= g\int_z^\infty \rho(z')\,dz
	\end{align}
	Note that it is okay to take the integral to infinity so long as we are dealing with a constant $\mathbf{g}$. This result suggests the definition of the \textbf{column density}:
	\begin{equation}
		\label{ippa.12} y(z)\equiv \int_z^\infty \rho(z)\,dz
	\end{equation}
	On the surface of the earth, the column density is approximately $y=1000\,\mathrm{g\,cm^{-3}}$. The column density is an important number (for us) to determine the details of heat transport.
	\subsection{Mean Molecular Weights}
	For an ideal gas, the total pressure of a mixed gas is simply
	\begin{equation}
		\label{mmw.1} P=\sum_{i=1}^N n_ikT
	\end{equation}
	The number density is compute via
	\begin{equation}
		\label{mmw.2} n_i=\frac{X_i\rho}{A_im_p}.
	\end{equation}
	Then the ion pressure is given by (assuming total ionization)
	\begin{equation}
		\label{mmw.3} P_{\mathrm{ion}}=kT\sum\frac{X_i\rho}{A_im_p}=\frac{kT\rho}{m_p}\sum\frac{X_i}{A_i}=\frac{kT\rho}{\mu_im_p}.
	\end{equation}
	For the electrons, we have
	\begin{equation}
		\label{mmw.4} P_e=n_ekT=kT\left(\sum Z_in_i\right)=\frac{kT\rho}{m_p}\sum\frac{Z_iX_i}{A_i}.
	\end{equation}
	Then the total pressure is just the sum of these two,
	\begin{equation}
		\label{mmw.5} P=P_{\mathrm{ion}}+P_e=\frac{\rho kT}{m_p}\left(\frac{1}{\mu_e}+\frac{1}{\mu_i}\right)
	\end{equation}
	So we define the overall mean molecular weight via
	\begin{equation}
		\frac{1}{\mu}\equiv \frac{1}{\mu_e}+\frac{1}{\mu_i}
	\end{equation}
%%%%%%%%%%%%%%%%%%%%%%
%  January 11, 2012  %
%%%%%%%%%%%%%%%%%%%%%%
	\textit{Wednesday, January 11, 2012}
	\subsection{Electric Fields in Stars}
	Recalling the scale height,
	\begin{equation}
		\label{efs.1} h=\frac{kT}{mg}
	\end{equation}
	we wish to find the scale height in a plasma of ionized hydrogen. In this plasma, we have $n_p=n_e$ due to electric neutrality. Then the overall pressure in this hydrogen plasma is
	\begin{equation}
		\label{efs.2} P=2n_pkT
	\end{equation}
	Using hydrostatic equilibrium, we get
	\begin{equation}
		\label{efs.3} 2kT\frac{dn_p}{dz}=-m_pn_pg
	\end{equation}
	which in turn gives us the differential equation
	\begin{equation}
		\label{efs.4} \frac{d\ln n_p}{dz}=-\frac{M_pg}{2kT}
	\end{equation}
	Which gives us a scale height of
	\begin{equation}
		\label{efs.5} h=\frac{2kT}{mg}
	\end{equation}
	We need to look at both plasmas separately. Incorporating the electric field for electrons, we have
	\begin{equation}
		\label{efs.6} \frac{1}{n_e}\frac{dP_e}{dz}=-m_eg-eE
	\end{equation}
	Likewise fo rhte protons,
	\begin{equation}
		\label{efs.7} \frac{1}{n_p}\frac{dP_p}{dz}=-m_pg+eE
	\end{equation}
	Now adding \eqref{efs.6} and \eqref{efs.7}, we recover hydrostatic balance. However, subtracting the two equations will get us the electric field:
	\begin{equation}
		\label{efs.8} 0=-m_eg+m_pg-2eE
	\end{equation}
	Which gives us the electric field being
	\begin{equation}
		\label{efs.9} \boxed{eE=\frac{1}{2}\left(m_p-m_e\right)g}
	\end{equation}
	This electric field is important to know what happens to a new particle introduced to the system. The gravitational force will be at odds with the electric field in determining the motion of such a particle. Additionally this explains how there is one scale height for a plasma as opposed to one for electrons and one for protons.
	\subsection{Self-Gravitating Objects}
	So far we have only considered systems where the acceleration due to gravity is constant. In any self-gravitating object, this is obviously not true. We will, however, continue to assume that such objects do not rotate. Additionally, we will be ignoring mass loss. Essentially all we must write down are equations of mass conservation, momentum conservation, and energy conservation. We'll start with momentum conservation.
	\subsubsection{Momentum Conservation and the Free-Fall Timescale}
	The momentum equation for a fluid is just
	\begin{equation}
		\label{mc.1} \rho\frac{d\mathbf{v}}{dt}=\rho\mathbf{g}-\bm{\nabla}P
	\end{equation}
	This equation essentially states that a self-gravitating object is neither collapsing nor expanding. If we were to ``shut off'' gravity or the pressure gradient, the star would either explode or collapse, respectively. Such a collapse would occur on the \textbf{free-fall timescale}, which we will now derive. Taking the pressure gradient out of \eqref{mc.1}, we retrieve
	\begin{equation}
		\label{mc.2} \mathbf{g}=-\frac{Gm(r)}{r^2}\hat{r}
	\end{equation}
	For this derivation, we will be using a \textbf{Lagrangian coordinate systems}. This is a system where the coordinates follow a particular fluid element. In essence, we are making the substitution
	\begin{equation}
		\label{mc.3} \frac{d}{t}\to\frac{\partial}{\partial t}+\mathbf{v}\cdot\bm{\nabla}
	\end{equation}
	Returning back to the derivation, \eqref{mc.2} gives us
	\begin{equation}
		\label{mc.4} \frac{dv_r}{dt}=-\frac{Gm(r)}{r^2}
	\end{equation}
	Initially, we have $t=0$, $v_r=0$, and $r=r_0$ with the radial velocity given by $v_r=dr/dt$. Then our differential equation is
	\begin{equation}
		\label{mc.5} \frac{d^2r}{dt^2}=-\frac{Gm}{r^2}
	\end{equation}
	As an order of magnitude estimate, this gives us
	\begin{equation}
		\label{mc.6}\frac{r}{t_{\mathrm{ff}}^2}\sim\frac{Gm}{r^2}\quad \Rightarrow \quad t_{\mathrm{ff}}^2\sim\frac{1}{Gm/r^3}
	\end{equation}
	So we define the free-fall timescale to be
	\begin{equation}
		\label{mc.7} t_{\mathrm{ff}}=\frac{1}{\sqrt{G\rho}}
	\end{equation}
	This is also the same as the Keplerian orbital period, modulo some uninteresting constants. The punchline of this whole argument is that a star that is \emph{not} in hydrostatic balance will respond on a timescale of the free-fall timescale. From this alone, we may conclude that the sun (and all other stars not currently exploding) is in hydrostatic balance. We will then assume that all stars are always in hydrostatic balance.\\
	
	\subsubsection{Repercussions for Stellar Structure}
	Stars are held up by gas pressure, radiation pressure, or both. The pressure gradients are what will be the ``restoring forces'' against gravity for our cases. In spherical symmetry, hydrostatic balance tells us
	\begin{equation}
		\label{rss.1} \frac{dP}{dr}=-\rho\frac{Gm(r)}{r^2}
	\end{equation}
	We will use this to derive the \textbf{Virial Theorem}, which relates the potential energy to the kinetic energy of a system. The equation of mass conservation states that
	\begin{equation}
		\label{rss.2} dm=4\pi r^2\rho(r)\,dr
	\end{equation}
	Now we multiply both sides of \eqref{rss.1} by $4\pi r^3\,dr$:
	\begin{align}
		\label{rss.3} \int 4\pi r^3\,dP&=-\int \rho\frac{G}{r^2}4\pi r^3\,dr m(r)\\
		\label{rss.4} &= -\int\frac{Gm(r)dm}{r} = E_{\mathrm{GR}}
	\end{align}
	Performing a similar analysis to the left side of \eqref{rss.3} gives
	\begin{align}
		\label{rss.5} \int 4\pi r^3\,dr\frac{dP}{dr} &= \left.4\pi r^2P\right|_{0}^R-3\left[4\pi\int Pr^2\,dr\right]\\
		\label{rss.6} &= -3\int P4\pi  r^2\,dr\\
		\label{rss.7}&=-3\avg{P}V
	\end{align}
	where we've defined the average pressure to be the pressure averaged over volume. Then the virial theorem tells us that
	\begin{equation}
		\label{rss.8}\boxed{\avg{P}=-\frac{1}{3}\frac{E_{\mathrm{GR}}}{V}}
	\end{equation}
	Now we examine the total energy:
	\begin{equation}
		\label{rss.9} E_{\mathrm{tot}}=E_{\mathrm{GR}}+E_{\mathrm{KE}}=-3\avg{P}V+E_{\mathrm{KE}}
	\end{equation}
	We need only relate the kinetic energy to the pressure to finish this equation off. For an ideal gas, we know that $P=NkT/V$, so the kinetic energy is $E_{\mathrm{KE}}=\frac{3}{2}NkT=\frac{3}{2}PV$. This gives a total energy of
	\begin{equation}
		\label{rss.10} E_{\mathrm{tot}}=-3\avg{P}V+\frac{3}{2}\avg{P}V=-E_{\mathrm{KE}}
	\end{equation}
	However for radiation, pressure is given by $P=\frac{1}{3}aT^4$ and $E/V=aT^4$. Taking this to its conclusion gives us
	\begin{equation}
		\label{rss.11} E_{\mathrm{tot}}\to 0\ \textrm{as the particles become relativistic}
	\end{equation}
	The origin of this result is in the momentum-energy relation of relativistic particles and non-relativistic particles. That is, $E=pc$ for ultra-relativistic particles and $E=p^2/2m$ for non-relativistic particles.\\
	
	\n The limiting energy of ultra-relativistic stars puts an upper level on the mass of large stars, since a total energy of a star being zero means unbinding the star (this is level of radiation is known as the \textbf{Eddington Limit}). In the ``normal case'' of an ideal gas star, the more traditional form of the virial theorem applies:
	\begin{equation}
		\label{rss.12} \frac{E_{\mathrm{KE}}}{\mathrm{particle}}\sim\frac{GM}{R}
	\end{equation}
	This is why stars typically behave with a negative heat capacity. That is, as a star radiates, $E_{\mathrm{tot}}$ is more negative, meaning that $R$ must decrease and the temperature $T$ (essentially the kinetic energy per particle) rises.
	\subsubsection{Applications of the Virial Theorm}
	the gravitational energy of an object is typically given by
	\begin{equation}
		\label{avt.1}E_{\mathrm{GR}}\approx -\frac{GM^2}{R}
	\end{equation}
	Using the virial theorem, we have
	\begin{equation}
		\label{avt.2} -E_{\mathrm{GR}}=-EV\frac{N}{V}kT
	\end{equation}
	Or,
	\begin{equation}
		\label{avt.3} \frac{GM}{R}\left(Nm_P\right)\sim 3NkT
	\end{equation}
	So we have
	\begin{equation}
		\label{avt.4} \boxed{kT\sim \frac{GMm_p}{R}}
	\end{equation}
	This temperature is the temperature of most of the material and is $T\sim T_c\sim \mathrm{core}$. For the sun, we then have $T\sim 10^7\ \mathrm{K}$. Interestingly, assuming hydrostatic equilibrium was all we needed to get a rough estimate of the sun's core temperature! One might note, though, that the surface temperature is significantly lower than the core temperature, so we must assume that there is heat loss taking place in the sun. Today the luminosity of the sun is
	\begin{equation}
		\label{avt.5} L_\odot = 4\times 10^{33}\ \mathrm{erg/s}
	\end{equation}
	If we assume there is no energy source for the sun other than gravitational energy, we can come up with a timescale (called the \textbf{Kelvin-Helmholtz timescale})
	\begin{equation}
		\label{avt.6} t_{\mathrm{KH}}=\frac{E_{\mathrm{GR}}}{L}\approx 3\times 10^7\ \mathrm{years}
	\end{equation}
	for the sun. This has been known for awhile and since the Earth is known to have existed much longer than $t_{\mathrm{KH}}$, scientists deduced that another energy source within the sun was needed to explain its longevity. We now know that this energy source is, of course, fusion. Note that at the center of the sun, the temperature of $10^7\ \mathrm{K}$ corresponds to an energy per particle of about 1 keV. The binding energy of helium is approximately 7MeV, approximately 7000 times bigger than the thermal content. Thus, the sun could last approximately 7000 times longer, bringing the projected lifetime of the sun up to a more reasonable (but still wrong) number of about 200 billion years.
\end{document}








